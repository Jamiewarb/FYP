\chapter{Usability Study}

\section{Tasks Performed} \label{usabilitystudytasks}
\begin{itemize}
\item \textbf{Task 1}: Find a HTML5 template snippet
\item \textbf{Task 2}: Store some specific code in a new public snippet
\item \textbf{Task 3}: Find a previously written snippet, update its title and set it to public
\item \textbf{Task 4}: Find a PHP snippet that's outdated, turn it into a collection and add an updated version to it
\item \textbf{Task 5}: Find a snippet you changed earlier, and revert the changes
\item \textbf{Task 6}: Perform interaction with snippets, including commenting, rating and edit requests
\end{itemize}

\section{Study Transcript} \label{usabilitystudytrans}

\begin{lstlisting}[caption={Transcription of Usability Study}]
\end{lstlisting}
J: Researcher (Jamie) \\
P: Participant

*The participant has the aim and deliverable of this dissertation explained to them. The computer in front has an open browser, with Shnip It open in the browser.*

J: \-\hspace{1.4cm} So let’s start by getting a feel of reusing code as a new user. Try to find a code snippet for a HTML5 document template. \\
P: Okay. \\
P: I can’t see any HTML snippets on the homepage, they look to be CSS and java mostly. I’ll click up here and search for it.  \\
P: *Clicks in search bar* \\
( Participant is able to quickly digest snippet information, including what language each snippet is, without having seen it before. )

P: *Types HTML5 boilerplate*  \\
*Page refreshes with search results*  \\
P: So this first one is called Cheeky HTML 5 boilerplate, and has more upvotes than any of the ones below it, so I'll pick this one. \\
P: *Click the name of the snippet* \\
( Expert performance, with no prompting. Understood snippet meta data, searching and accessing the snippet, immediately. )

*The snippet page opens to display the snippet*

P: Oh I like that the description looks like a html comment. Does that happen for each language? \\
J: \-\hspace{1.4cm}No, it's entered by the user when they type the snippet, so it can just be text. \\
P: You could definitely code that yourself! 

J: \-\hspace{1.4cm}Okay now you can find snippets, let's try adding your own one. Here's some code that you want to store  \\
J: \-\hspace{1.4cm}*Points to a wordpad file open on the computer*.  \\
J: \-\hspace{1.4cm}Try to create a snippet with appropriate meta data, that's public for everyone to use. \\
P: Okay so there's a create button in the navbar  \\
P: *Clicks create, then clicks A Snippet*.  \\
*The login page appears* \\
P: Oh I need to login first. Do you have a test account I can use or should I make one? \\
( For the purpose of this study being quick, the researcher logs in using a pre-existing account )

*The create snippet page appears*

P: I like that login screen, the bar of colours looks nice. \\
P: Okay so, first I can set the language, so this snippet looks like CSS to me, yeah? \\
J: \-\hspace{1.4cm}Yeah \\
P: So I'll select CSS. What happens if I want to store a snippet that- oh there's an Other selection. So you could pick that if it's not supported? Cool.  \\
P: Then I'll enter some tag words. 
P: *Types a few key words (background, color, gradient)*  \\
P: *Types a title (Set background colour gradient)* \\
P: So in the description, I have to write the comment tags myself to make it look nice on the snippet? \\
J: \-\hspace{1.4cm}Yep, they're not automatic \\
P: Okay so I'll put those in  \\
P: *Types a description, inside of /* and */ tags* \\
P: And finally the snippet goes here  \\
P: *Pastes snippet, and clicks Create Public Snippet* \\
( Expert performance again, easily finding the create section and submitting the snippet. ) \\
*The snippet is created and displayed on screen*

J: \-\hspace{1.4cm}Okay great, well done! So now, let's say you've decided a previous snippet you've written needs to be updated, and then needs setting to public. Try to find the private snippet called 'Not a good snippet', and change it to 'Good Snippet', and set it public. And don't use the search bar for this bit either. \\
P: Find it? Okay so there's a place called Boards, I'm guessing that's like a collection of snippets.  \\
( User understands the terminology of the site without it being mentioned to them ) \\
P: I'll go there and see if mine are in there. I don't know where I'm looking though.  \\
P: *Clicks Boards in the navigation bar. More options appear.* \\
P: *Clicks My Boards* \\
*A list of the user's boards appear* \\
P: So there's quite a lot of boards here, I'm not sure where to start without the search bar. Is there a way to search through this My Boards page? \\
J: \-\hspace{1.4cm}No but I like that idea, I think that's good to limit the search to specific spaces. However, advanced search and sort will come later in the project as it's not in the scope of my dissertation. \\
P: Okay, so do I just need to go through them all? \\
J: \-\hspace{1.4cm}Try going to your profile page. \\
P: Profile page... So up here  \\
P: *Clicks on account name, then on View Profile* \\
*The user's profile page opens* \\
P: So there's some snippets on the right, I'm guessing I should click View All... *Clicks View All* \\
*The page changes to show all the user's snippets, sorted by most recent updated date* \\
P: And then there's two pages of snippets, so that's easier.  \\
P: *Presses Ctrl + F, types Not a, and then clicks on the 'Not a Good Snippet'* \\
*The appropriate snippet page is displayed on screen* \\
P: Got it! \\
( We hadn't been to the profile page yet, but once there, the user understood immediately that they could view all of their snippets from there. )

P: So I need to change the title, and the visibility setting to public? \\
P: *Clicks pencil icon for edit* \\
*The edit snippet page is displayed* \\
( No prompting here! User understands the function icons and what they will do ) \\
P: *Changes the title as requested, and sets the snippet to public* \\
( Again no prompting on this page! It closely resembles the Create Snippet page in layout and design, and this is seen by the expert performance demonstrated! ) \\
P: Okay, I think that's that. Just the title and the publicity setting? \\
J: \-\hspace{1.4cm}Yep, that's it! Perfect. \\
P: *Clicks save snippets* \\
*The snippet page is displayed again*

J: \-\hspace{1.4cm}So let's do something a bit more complex now - there's a public PHP snippet called 'Access Database', but it's written in PHP 5. We want to change it to PHP 7, without overriding the existing snippet. I want you to find the snippet, then turn it into a Collection, and add a second snippet to it, the PHP 7 one. \\
P: What's a collection? \\
J: \-\hspace{1.4cm}It's where one 'snippet', so to speak, actually holds multiple snippets, like a snippet for a button might have html, css and javascript. \\
P: Okay I see. \\
P: *Clicks Boards > Public Boards* \\
*List of public boards is displayed* \\
( User seems to be under the impression that we still don't want them to use the search bar, which suits our needs fine ) \\
( User immediately understands where to find the snippet, without having been there before, due to the Boards > My Boards ) \\
J: \-\hspace{1.4cm}What made you click here? \\
P: You said it was public, and I'm not using the search bar, so I figured I would find it here. I'm guessing it's in this PHP board here \\
P: *Clicks to open the PHP board* \\
*The PHP board is displayed on screen, showing some of the snippets* \\
P: *Presses Ctrl + f to find Access Database, and clicks the title* \\
*The snippet is now displayed on screen* \\
P: Um, oh yeah okay this is it. Yeah I thought this wasn't PHP 5 but it is. Do you want me to write one in 7? Or do you have another one? \\
J: \-\hspace{1.4cm}Yeah I have one *Points to another wordpad file* \\
P: Okay so, I click Edit snippet, and then Create Collection? \\
J: \-\hspace{1.4cm}Yep \\
P: *Clicks Create Collection* \\
*The edit snippet page changes to show it's now in a collection, and extra options appear* \\
P: And so now I have a snippet in the collection so I need to add another one  \\
P: *Clicks the + button to add another snippet* \\
( User immediately grasps the concept of Collections and how to build them from snippets ) \\
P: Which wordpad file was it? \\
J: \-\hspace{1.4cm}*Points to the file* This one \\
P: Okay thanks, so I'll paste all this in to the new boxes. It's good because it's all the same fields and same layout and everything so it seems quite easy to make more snippets. I'd get used to this quite quickly. Just need better searching so I can find things faster. \\
J: \-\hspace{1.4cm}Yeah, like I said that will come in the future, otherwise it's too much for one dissertation. \\

................ 

P: *Enters the rest of the snippet detail then hits save* \\
*Page changes to view the snippet, with a message saying it's awaiting confirmation, and a message explaining why* \\
P: Okay, I've added the PHP 7.0 one. It says that it's awaiting confirmation. \\
J: \-\hspace{1.4cm}Yeah so there's two ways for this to be  added - either the original author accepts the change request he just received, or the community vote it positively past a threshold. Currently that threshold is 20 positive votes, but the plan is to make it dynamic based on current and historical data trends, and user weightings. Unfortunately, again, that's out of scope for the system at the moment. \\
P: Great so, if the user accepts it, it just changes? Can it go back? \\
J: \-\hspace{1.4cm}Yeah, all snippets have complete version history and control. 

J: \-\hspace{1.4cm}So while we're on that topic, lets go back and revert the 'Good Snippet' back to the previous 'Not a good snippet', making sure it's private again. \\
P: Okay. *Clicks on profile, finds snippet, clicks for version history* \\
( User is able to easily remember and recall previous actions like an expert user, and navigates perfectly and efficiently. ) \\
*The snippet is displayed, then the history of the snippet is displayed as a timeline* \\
P: Oh so there's a timeline a bit like github, or stack overflow. You should have diff to show exactly what changed. \\
P: *Clicks the restore snippet button*. \\
P: And so that's it. Does it save the 'Good Snippet' now in the version history? \\
( User intuitively navigates and interacts with the version control without prompting or prior knowledge! ) \\
J: \-\hspace{1.4cm}Yeah, it makes a new record. I've toyed with just moving the order when you restore, or deleting the HEAD up to that snippet, but I think this way is more complete, though it might need some maintenance for long history snippets. I might add an option to delete history, or hide from the main listing or something like that.

J: \-\hspace{1.4cm}Finally, lets do some interaction. From the homepage, take a look through some of the snippets and use the interaction functions to express your opinions. \\
P: Interaction functions? \\
J: \-\hspace{1.4cm}Like leaving a comment, favouriting a snippet, rati- \\
P: Ah okay I see what you mean.  \\
J: \-\hspace{1.4cm}-ting a snippet. \\
P: *Clicks Shnip It logo to go to homepage* \\
*Home page appears* \\
P: I'll just read a few then and react as I would do. \\
P: *Participant begins going through snippets, and voting them up or down and commenting why. The participant submitted an edit request for one too* \\
( All interaction was unprompted, and the participant clearly understood what he was doing, without making any mistakes. )

J: \-\hspace{1.4cm}Okay I think that's enough, thanks. 

---------------

J: \-\hspace{1.4cm}So that's all of the test cases, so thank you for your time! \\
P: Yeah no worries, it looks good \\
J: \-\hspace{1.4cm}So I just have a few questions if you wouldn't mind answering them. \\
P: That's fine, go ahead 

J: \-\hspace{1.4cm}Okay, so, first of all, what are your first impressions of the system? What are your thoughts? \\
P: Well it seems to do what it sets out to do, albeit it needs more search to be complete, but I know you have that on the roadmap. It looks clean which is good, and although each snippet has a lot of information, it's clearly laid out and well styled to differentiate the data. \\
I wonder how you will sort the 'popular' snippets on the homepage, as right now it looks like they're just being chosen based on the most votes, which would lead to the most popular snippets being permanently sorted to the top with nothing else able to reach it. Though I suppose that may be desireable. I'll let you work that one out! \\
Overall I quite like it and would like to follow its development.

J: \-\hspace{1.4cm}Did you feel comfortable using the system without prompting? \\
P: Yeah I don't think you prompted me very much, other than with going to the profile page, but I couldn't use the search bar so that makes sense. For the most part, it was quite obvious what each part of the system was going to do.  \\
J: \-\hspace{1.4cm}Can you expand on what you mean by how the parts were obvious? \\
P: So like, the call to action buttons did what I thought they would do, because the text and icons on those are similar to other website's that I've used. I didn't have to guess much since it's a similar feel to other systems.  \\
J: \-\hspace{1.4cm}And did you feel it was quite simple and clear, or did it feel cluttered or busy? \\
P: Yeah I like the layout and the design, it didn't feel like there was too much or anything. There's a lot of information on the snippet when they're in a list, but the styling on the text makes it easy to distinguish what each piece of data is, and you use colour well to emphasize the important bits, so I would say yeah overall it is simple and clear. 


J: \-\hspace{1.4cm}Do you feel the system is suitable for small scale developers to utilise for code reuse? \\
P: Yes I think it would be, as it would allow a central repository for common, uh, snippets. I wonder if splitting it into different sub-websites, like Stack Overflow has done, for example one for web developers, one for other mediums etc, but I don't know the scope of that. You could also include other files like common designs, say for UML diagrams, or README files and things like that rather than just code. \\
I think it would be good for web developers though, for sure, I really like it, especially the collaboration part of it, if it works well.

J: \-\hspace{1.4cm}And how do you feel then about using it with your company to share and reuse code, such as having several private boards for your colleagues to use? \\
P: So like giving each team a board, and then giving all employees access to their team boards, or even access to all the boards? I think it would work quite well, but I wouldn't want to have to navigate through all the public stuff as well. I would want it to be maybe switched into a company mode, maybe -company name redacted-.shnip.it as the URL, then all the searching is just within that companies assets etc. But I think it would work well company wide too. Perhaps without the rating system though, I don't know if that's good or bad in a workplace environment. You might need to look in to that at a deeper level.


J: \-\hspace{1.4cm}Do you feel having the collaborative elements of the website would help keep the code updated with language advancements? \\
P: So this is like in reference to the PHP 7 snippet that we did? Yeah I think it's something I've found annoying when I've had to switch to a new branch of PHP, and I sort of just start a new snippet collection because I don't have the time to go through and update them all. If people contribute to this and we have a central repository of generic snippets in the different code versions, yeah I think that's something that the website could do well. It certainly won't harm the ability to keep the code updated, but I think maybe some sort of moderation would be required, similar to Stack Overflow - maybe power users that are given extra power to moderate submissions, like a community to govern the community.

J: \-\hspace{1.4cm}And did you feel the collaborative elements of the website made it better than conventional storage systems?
P: I think the trade off for having the collaborativity is the loss of a personal storage - it's now shared with others, and with that brings other coding stylings and opinions. That being said, you're not removing the personal storage element as you have your private boards, so it's more like choosing which route you want to go down.  \\
P: So yeah I would say it is better than the conventional storage, as you can choose to have the simple, personal boards, or to contribute and collaborate with the wider community, which lowers your workload considerably, expands your options and provides you with, what I assume you'll strive for, the best possible code snippets.

J: \-\hspace{1.4cm}Do you think such a system would increase the quality of reuseable code throughout the community in comparison with a non-collaborative system? \\
P: Stack Overflow is usually good at weeding out the bad answers and highlighting the good answers, at least in my experience, but any sort of collaborative system is only going to be as good as its users. I think the idea of super users as moderators, who have shown their expertise, would be the key to keeping the quality of the reuseable code high, but the best coder in the world is always going to produce a better personal code repository than 500 users all new to programming.  \\
P: I think, therefore, providing some form of moderation is the key to how successful this quality element of collaboration is, and also being able to grab and keep the attention of good programmers, and fold them into the community, is what will make it good. So yeah, providing you can attract and retain those expert level programmers, I think it will increase the quality of reuseable code.

J: \-\hspace{1.4cm}Great, thank you, that's all of my questions \\
P: Okay cool. Oh send me a link to this too, I want to keep an eye on it. \\
J: \-\hspace{1.4cm}Yeah for sure, I'll email you when it goes live

...............


